% !TEX encoding = UTF-8
% !TEX program = lualatex

\documentclass[a4paper]{article}

\usepackage[left=1.5cm, right=1.5cm, top=1cm, bottom=2cm]{geometry}

\usepackage{mathtools, amssymb}

\usepackage{emoji}
    \setmainfont{Noto Serif}
    \setsansfont{Noto Sans}
    \setmonofont{Noto Sans Mono}

\usepackage{tikz}

\tikzset{
    every picture/.style={baseline=(current bounding box)},
    code block/.style={
        fill=gray!10, align=left, inner ysep=0pt,
        font={
            \hskip-20pt
            \ttfamily
            \catcode32=12 \catcode13=13
        }
    },
    code block right/.style={
        overlay, shift={(18, 0.3)}, 
        every node/.style={code block, below left}
    }
}
\newcounter{p}
\def\Problem{
    \stepcounter{p}[\arabic{p}]
    \hskip0pt plus1em
    \immediate\write\answer{\hskip0pt plus1em[\arabic{p}]}%
}
\newcounter{c}
\def\yes{\no\writeanswer{\Alph c}}
\def\no{%
    \hskip0pt plus10em
    \stepcounter c\ifnum\value c=27\setcounter c1\fi(\Alph c)\nobreak\ 
}
\def\writeanswer#1{\immediate\write\answer{
    \noexpand\tikz\noexpand\node
    {\noexpand\fontsize{24pt}{0}\fontspec{Comic Sans MS}#1};
}}
\def\blank#1{
    \underline{\phantom{#1}}
    \writeanswer{#1}%
}
\newwrite\answer
\immediate\openout\answer=\jobname.answer

\begin{document}

\parskip 1em
\parindent 0pt

\catcode`\#=12
\catcode`\%=12
\catcode`\^=12
\catcode`\&=12
\catcode`\_=12
\catcode`\'=13 \def'{\rmfamily}
\catcode`\`=13 \def`{\ttfamily}
\catcode13=13 \def
{\\} \catcode13=5\relax

\fontspec{Noto Serif TC}

\section*{Computer Programming - Midterm Mock Exam - Question Sheet}

考試時間 Test time 15:50 pm to 18:10 pm; 2.5 hours.
以教室後方時鐘為準 Using the clock at the back of the classroom as official time.
在答案卷上作答 Answer on the answer sheet.
每題一分 One point per problem.
全對才給分 No partial credit.
不倒扣 No penalty for wrong answers.
禁止外部資源(比照學測、分科) No external resources, like TOEFL.

\rmfamily

\section{Program Meta}

\tikz [code block right] \node {
    8888888888888888888888888888888888888888
    8888888888888888888889888888888888888888
    8888888888888888888888888888888888888888
    8888888888888888888888888988888888888888
    8888888888888888888888888888888888888888
    8888888888888888888888888888888888888888
    8888888888888888888888888888888888888888
    8888888888888888888888888888888888888888
    8838888888888888888888888888888888888888
    8888888888888888888888888888888888888888
};

\Problem Find a hidden `9'.

\Problem Find another hidden `9'.

\Problem Find the other hidden number.

\hangindent-10cm \hangafter-9
\Problem C is invented in
\yes 1972
\no 1982
\no 1992
\no 2002

\hangindent-10cm \hangafter-9
\Problem C++ is invented in
\no 1953
\no 1963
\no 1973
\yes 1983

\tikz [code block right] \node {
    \#include <iostream>
    using namespace std;
    int main() \{
        cout << "hello"
             << "ntu";
    \}
};

\hangindent-5cm \hangafter-9
\Problem The code to the right outputs
\no `he11ontu'
\no `hell0ntu'
\yes `hellontu'

\hangindent-5cm \hangafter-9
\Problem What if we `cout << "hello" "ntu";'?
Select two correct answers.
\no It is a syntax error.
\yes It outputs `hellontu'.
\no It outputs `hello ntu'.
\yes String literals are concatenated.

\hangindent-5cm \hangafter-1
\Problem The purpose of using namespace `std' is to
\no follow the C++26 standard for variable names
\yes access the objects without repeating the prefix
\no follow the latest C++ standard for naming conventions

\Problem For a beginner who just finished the `hello world' program,
a good next step is to write
\no a `hello world' program in python and cython
\yes a program that takes two strings and prints them
\no a program that can draw Hilbert curve using turtle

\Problem `#include <iostream>' includes
\no the entire C++ standard library
\no the implementation of `cout' and `cin'
\yes a header file that help you read from keyboard
\no the interface to file-in and file-out operations

\tikz [code block right] \node {
    int main (int argc, char *argv[]) \{
        return 0;
    \}
};

\hangindent-8cm \hangafter-3
\Problem Which two are correct about `main'?
\no It should always return `0'.
\no It is the only function allowed in a C++ program.
\yes It is special in that you cannot call it recursively.
\yes For `g++-14', `argv' may contain things like `-o' and `-Wall'.

\tikz [code block right] \node {
    g++-14 -Ofast -Wall option_pricing.c -o op
    op <<< "100 0.05 0.2 1 100"
};

\hangindent-10cm \hangafter-2
\Problem Which one is incorrect regarding these terminal commands?
\no `-o op' specifies the output file name.
\no `g++-14' refers to the compiler developed by GNU.
\no `-Wall' turns on some warnings that sometimes helps finding bugs.
\yes `option_pricing.c' is a C file, so `g++' cannot possibly compile it.
\no `-Ofast' is an optimization flag; it tries to make the program faster.

\Problem Same as above, choose all incorrect statements.
\yes `<<<' suggests that `op' will `cout' something.
\no To post-process the output of `op', we can use `|'
\no If something goes wrong, use `code option_pricing.c' to edit.
\yes `100 0.05 0.2 1 100' will become `argv' in the `main' function
\no If `<<<' is replaced by `<<', it suggests that `op' will `cin' something.

\Problem What command line tools outputs
`00000000  cf fa ed fe 07 00 00 01  03 00 00 00 02 00 00 00  |................|'
\no cat
\no vim
\no touch
\yes hexdump

\Problem What does the command line tool `grep' do?
\yes It searches for matching text.
\no It visualizes data with ASCII art.
\no It compiles `Go representation' code.
\no It grabs and prints the environment variables.

\Problem What does command line command `cd' stands for?
\no clear display
\yes change directory
\no create dictionary
\no compile definition

\section{Variables and Types}

\Problem A shadowed variable
\no is not properly initialized
\no is highlighted in the text editor
\yes shares the same name with another
\no is never highlighted in the text editor
\no is an implicit variable created by the compiler

\Problem Which are valid variable names?
\yes `variablename'
\yes `VariableName'
\no `*variablename'
\yes `_variablename'
\no `0variablename'

\Problem What is `date' after `float date = 2025 / 10 / 15;'?
\blank{13}

\Problem What is the range of $32$-bit signed integer?
\no $0$ to $2\sp{31}$
\no $0$ to $2\sp{32}$
\no $0$ to $2\sp{32} - 1$
\no $-2\sp{31}$ to $2\sp{31}$
\yes $-2\sp{31}$ to $2\sp{31} - 1$

\Problem What is the largest `n' such that
`signed char fibonacci(int n)' does not overflow,
with the initial condition that `fibonacci(2)' is `2'.
\no 9
\yes 10
\no 11
\no 12

\Problem Regarding the lengths of variable names,
what should we, the programmer, keep in mind?
\no Longer names makes the compiler slower.
\yes Longer names makes the program clearer.
\no Shorter names makes the compiler safer.
\no Shorter names makes the program faster.  

\Problem \verb"cout << int('c') << int('b');" outputs `9998'.\\
Accordingly, \verb"cout << int('x') << int('y') << int('z');"
outputs \blank{120121122}

\Problem \verb"cout << int('X') << int('Y') << int('Z');"
outputs \blank{888990}

\Problem What is camel case?
\no A `switch' statement without `break'.
\no The default branch of an `if' statement.
\yes Capitalizing the first letter of each word.
\no A functional wrapper protecting camel functions.

\Problem The binary representation of `int16_t a = 1111' is
\blank{0000010001010111}

\Problem The binary representation of `int16_t a = -12' is
\blank{1111111111110011}

\Problem How many bytes of memory does `int16_t A[100000];' need?
\no 100000
\yes 200000
\no 400000
\no 800000
\no 1600000

\Problem `float' is
\no always 16-bits
\no sometimes 16-bits
\yes at least 32-bits
\no at most 32-bits

\tikz [code block right] \node {
    00111111100000000000000000000000 = 1
    01000000000000000000000000000000 = 2
    01000000010000000000000000000000 = 3
    01000000100000000000000000000000 = 4
    01000000101000000000000000000000 = 5
    01000000110000000000000000000000 = 6
    01000000111000000000000000000000 = 7
    01000001000000000000000000000000 = 8
    01000001000100000000000000000000 = 9
    01000001001000000000000000000000 = 10
};

\hangindent-9cm \hangafter-9
\Problem To the right are some floating-point numbers
and their binary representations.
What standard is that?
\yes IEEE 754
\no IEEE 768
\no IEEE 3.14
\no IEEE 802.11

\hangindent-9cm \hangafter-9
\Problem How much precision is that?
\no half-precision
\yes single-precision
\no double-precision
\no quadruple-precision

\hangindent-10cm \hangafter-9
\Problem What is the binary representation of `64'?
\blank{0 10000101 00000000000000000000000}

\Problem What is the binary representation of `-0.0625'?
\blank{1 01111011 00000000000000000000000}

\Problem What is the binary representation of `3.5'?
\blank{0 10000000 11000000000000000000000}

\Problem In the range of `int', `1010 + 1020 + 1030 + 1040' is
\no `5000'
\no `6174'
\yes `4100'
\no overflow

\Problem Which of the following is the most common way to define constant `PI'
to the precision of 32-bit `float`?
\no `float PI = 3.14;'
\yes `float PI = 3.1415927;'
\no `float PI = 3.1415926535 8979323846;'
\no `float PI = 3.
1415926535
8979323846
2643383279
5028841971
6939937510
5820974944
5923078164
0628620899
8628034825
3421170679;'

\section{Operators and Arithmetic}

\Problem What is `3 << 3'?
\blank{24}

\Problem What is \verb"'x' ^ 'X'"?
\no 0
\no 1
\no 2
\no 4
\no 8
\no 16
\yes 32
\no 64
\no 128
\no 256

\Problem How to check if `n' is a multiple of 17?
\no `n / 17 == 0'
\yes `n % 17 == 0'
\no `n // 17 == 0'
\no `n %% 17 == 0'

\Problem What happens after `b += a;'?
\no `a' increases by `b'
\yes `b' increases by `a'
\no `a' becomes positive `b'
\no `b' becomes positive `a'

\Problem If `int a = 3, b = 2;', select all that compile.
\yes `cout << a + b';
\yes `cout << a + + b';
\yes `cout << a ++ + b';
\yes `cout << a + ++ b';
\yes `cout << a ++ + + b';
\yes `cout << a + + ++ b';
\yes `cout << a ++ + ++ b';

\Problem What is the overall output of the choices above
(after removing choices that do not compile)?
\blank{55577910}

\tikz [code block right] \node {
    bool isSquare (int n) \{
        // your code here
        // your code here
        return false;
    \}
};

\hangindent-6cm \hangafter-9
\Problem Finish this function that checks whether `n' is a perfect square.
You can assume that `n' is between `0' and `100'. \\
\blank{if (n == int(sqrt(n)) * int(sqrt(n))) return true;}

\hangindent-6cm \hangafter-2
\Problem Use this function `int max3(int a, int b, int c);'
to express the medium of `int x, y, z;'.
(No sorting because it is not an array.)
\blank{x + y + z - max3(x, y, z) + max3(-x, -y, -z);}

\tikz [code block right] \node {
    cout << "Height centimeters?";
    cin >> height;
    cout << "Weight in kilograms?";
    cin >> weight;
    cout << "BMI is " << /*your code here*/;
};

\hangindent-10cm \hangafter-9
\Problem Body mass index is the quotient of
weight in kilograms and height in meters squared.
Complete the code.
\blank{weight * 10000. / height / height}

\hangindent-10cm \hangafter-1
\Problem Select all that are true.
\no `pn -> n' makes `pn' a pointer to `n'.
\no `A => B' asserts that `B' is a derived class of `A'.
\yes `a - - b' is equal to `a + + b' for integers `a' and `b'.
\no `(a * b) * c' must equal to `a * (b * c)' for `float a, b, c'
if there is no overflowing or underflowing.

\Problem What is the most readable yet correct way to assign 4 percent of 25 to `neo'?
\no `float neo = 4\% * 25;'
\no `float neo = 25\% * 4;'
\no `float neo = 0.05 * 25;' 
\yes `float neo = 4 / 100. * 25;'

\section{Flow Control}

\tikz [code block right] \node {
    1 1 1 1
    1 2 3
    1 3
    1
};

\hangindent-3cm \hangafter-9
\Problem To the right is the Pascale triangle.
For a given input `n', output the first `n' rows of
the Pascal triangle in the same format.
Except that all numbers are `% 3';
\blank{for (...) \{for (...) \{...\}\}}

\hangindent-3cm \hangafter-1
\Problem Select true statements.
\yes `for(;;)' is an infinite loop.
\no `for(ture)' is an infinite loop.
\yes `while(1)' is an infinite loop.
\no `while(2)' cannot be stopped by `break'.

\Problem Select true statements.
\yes `break' will break the inner-most loop.
\no `continue' will restart the outer-most loop.
\no `fbreak' will break the inner-most for-loop.
\no `wbreak' will break the outer-most while-loop.

\tikz [code block right] \node {
    float f = 0;
    float dfdt = 1;
    float dt = 1e-3;
    for (float t = 0; t < 10; t += dt) \{
        float ddfddt = -f;
        dfdt += ddfddt * dt;
        f += dfdt * dt;
    \}
};

\hangindent-9cm \hangafter-9
\Problem Some numerical ODE code is presented to the right.
What is the equation being solved?
\no $f\sp{\prime} = f$
\no $f\sp{\prime\prime} = f$
\no $f\sp{\prime} = -f$
\yes $f\sp{\prime\prime} = -f$

\hangindent-9cm \hangafter-9
\Problem Select the impliedinitial conditions.
\no $f(1) = 0$
\yes $f(0) = 0$
\no $f\sp\prime(0) = 0$
\yes $f\sp\prime(0) = 1$

\hangindent-9cm \hangafter-2
\Problem Select all that apply.
\yes In industry, there are better methods.
\no `1e3' and `0x1e3' are of the same type.
\no The loop runs for about 100000 iterations.
\no `t' is casted into an `int' before comparing to `10'.

\tikz [code block right] \node {
    int S = 0;
    for (int i = 0; i < 100000000; i++) \{
        if (rand() < 0.003) \{ S++; \}
    \}
};

\hangindent-9cm \hangafter-9
\Problem After the loop to the right, with high probability,
`S' is closest to
\yes 300
\no 3000
\no 30000
\no 300000

\hangindent-9cm \hangafter-1
\Problem Recall that modern CPUs have frequency ranging from 1 to 5 GHz.
How long does it take to run the loop in the previous question?
(Assuming no flags like `-Ofast')
\no about 1--10 microseconds
\yes about 1--10 seconds
\no about 1--10 days
\no about 1--10 years


\section{Arrays}

\Problem What does `int A[10][20] = \{\};' do?  Select all that apply.
\yes It sets all 200 entries to zero.
\no It is faster than `int A[10][20]';
\yes It allocates some space for 200 integers.
\no It needs some space for 10 pointers to integers.

\Problem Declared as `int A[100];' the type of `A' is best described as
\no an array
\no an integer array
\no an array of length 100
\yes an array of 100 integers

\Problem Which one sums all elements of `A'?
\no \tikz [baseline=(current bounding box)] \node [code block] {
    for(i = 1; i <= 100; i++)
        sum += A[i];
};
\yes \tikz [baseline=(current bounding box)] \node [code block] {
    for(i = 99; i >= 0; --i)
        sum += A[i];
};
\no \tikz [baseline=(current bounding box)] \node [code block] {
    for(i = 1; i < 100; ++i)
        sum =+ A[i];
};

\tikz [code block right] \node {
    int A[10] = \{/*hidden*/\};
    int max0 = 0;
    for (int j = 0; j < 10; j++) \{
        max0 = max0 < A[j] ? max0 : A[j];
    \}
};

\hangindent-9cm \hangafter-9
\Problem When the loop ends, `max0' is
\yes `0' if all `A[j]' are positive
\no the maximum of an integer array `A'
\no `inf' if `A' contains negative numbers
\no undeclared if the array `A' is not initialized 

\hangindent-9cm \hangafter-2
\Problem When the loop ends, `j' is
\no 0
\no 9
\no 10
\yes not usable, as `j' is local to the loop

\Problem Joseph wants to solve heat equation 
$\displaystyle\frac{du}{dt} = \frac{d\sp2u}{dx\sp2} + \frac{d\sp2u}{dy\sp2}$
using finite element method.
What seems to be wrong with his code?
\begin{verbatim}
void EvolveOneSecond(float Temperature[100][100], float conductivity) {
    float Temporary[100][100];
    for (int x = 0; x < 100; x++) {
        for (int y = 0; y < 100; y++) {
            float ThisT = Temperature[x][y];
            float SumDiff = 0;
            SumDiff += Temperature[x-1][y] - ThisT;
            SumDiff += Temperature[x+1][y] - ThisT;
            SumDiff += Temperature[x][y-1] - ThisT;
            SumDiff += Temperature[x][y+1] - ThisT;
            Temporary[x][y] = SumDiff * conductivity;
        }
    }
    for (int x = 0; x < 100; x++) {
        for (int y = 0; y < 100; y++) {
            Temperature[x][y] = Temporary[x][y];
        }
    }
}
\end{verbatim}

\Problem There are two DNA sequences `char A[200], B[200];'.
Knowing that they came from a very old, therefore noisy, machine,
how do I check if the second half of `A' matches the first half of `B'?
\no find the longest common substring of `A' and `B'
\no find the shortest common supersequence of `A' and `B'
\no find the longest palindromic substring of `A[100..200] + B[0..100]'
\yes compute the Hamming/Levenshtein distance between `A[100..200]' and `B[0..100]'

\section{Functions}

\Problem Select the three scenarios where
your function needs to take another function as an argument.
\yes To sort an array
\no To reverse an array
\yes To time a function
\no To remove nans from an array
\no To define a recursive function
\yes To cache the values of a function

\Problem Using `a', `b', `c', `d', `x', `add(, )', and `mul(, )', instead of `+' and `*',
How to  express cubic polynomial $ax\sp3 + bx\sp2 + cx + d$?

\tikz [code block right] \node {
    float f(float x) \{ return 10 * x + 1; \}
    float f(int x) \{ return 10 * x + 2; \}
    int g(float x) \{ return 10 * x + 3; \}
    int g(int x) \{ return 10 * x + 4; \}
};

\hangindent-10cm \hangafter-9
\Problem According to the function definitions, what is `f(5)'?
\blank{52}

\Problem What is `g(1)'?
\blank{54}

\Problem What is `f(g(1))'?
\blank{542}

\Problem What is `f(f(g(g(f(1)))))'?
\blank{523421}

\tikz [code block right] \node {
    #include <iostream>
    #include <random>
    using namespace std;
    int main () \{
        mt19937 rand(20251015), rane(20251015), ranf(20251015);
        cout << rand() << " " << rand() << " " << rand() << endl;
        cout << rane() % 2 << endl;
        cout << ranf() % 3 << endl;
        cout << rane() % 2 << rane() % 2 << endl;
        cout << ranf() % 3 << ranf() % 3 << endl;
    \}
};

\hangindent-14cm \hangafter-9
\Problem The first line of the output is \hfil
`2357136044 2546248239 3071714933'.
What is the second line of output?
\blank{1}

\hangindent-14cm \hangafter-9
\Problem What is the third line of output?
\blank{2}

\hangindent-14cm \hangafter-9
\Problem What is the fourth line of output?
\blank{01}

\Problem What is the fifth line of output?
\blank{00}


\section{Pointers and Memory}

\Problem `int *a[10];' declares a length-10 array of pointers to integers.
How to declare a pointer to an array of 10 integers?
\blank{int (*a)[10];}

\Problem What is the correct way to declare two pointers, both pointing to integers?
\no `int* a, b;`
\yes `int *a, *b;`
\no `&int a, b;`
\no `int &a, &b;`

\tikz [code block right] \node {
    int a = 10;
    int *p = &a;
    *p += 20;
    cout << a;
};

\hangindent-4cm \hangafter-1
\Problem
The code to the right outputs
\no `10'
\no `20'
\yes `200'

\hangindent-4cm \hangafter-2
\Problem Why is `malloc(100 * sizeof(int))' preferred over `malloc(400)'?
\no Because it is faster to avoid computer `sizeof(int)'.
\yes Because `int' may be 4 bytes on some machines but 8 bytes on others.
\no Because `sizeof(int)' is 8, so `400' is half of what we actually need.
\no Because `sizeof(int)' is 2, so `400' is twice of what is actually needed.

\tikz [code block right] \node {
    int a = 1, b = 2, c = 3;
    int *p = &a, *q = &b, *r = &c;
    int **x = &p, **y = &q, **z = &r;
    cout << **x << **y << **z << endl;
    swap(x, y);
    cout << **x << **y << **z << endl;
    swap(a, c); swap(q, r);
    cout << **x << **y << **z << endl;
    swap(*x, *z);
    cout << **x << **y << **z << endl;
    swap(*p, *q); swap(**y, **z);
    cout << **x << **y << **z << endl;
};

\hangindent-8cm \hangafter-9
\Problem Look at the code to the right,
what is the first line of output?
\blank{123}

\Problem What is the second line of output?
\blank{213}

\Problem What is the third line of output?
\blank{132}

\Problem What is the fourth line of output?
\blank{231}

\Problem What is the fourth line of output?
\blank{312}

\Problem `1*a**p***x' is
\blank{3}

\section{Structures and Classes}

\tikz [code block right] \node {
    struct Circle \{ float x, y, r; \};
    struct Circle Scale (struct Circle c, float s) \{
        // your code here
        return big;
    \}
};

\hangindent-11cm \hangafter-9
\Problem Finish the function `Scale' such that 
`Scale(c, 2)' shares the same center with `c' but has double the radius.

\hangindent-11cm \hangafter-9
\Problem `RotateCCW' keeps the center of a rectangle
and rotate it by 90 degrees CCW.

\tikz [code block right] \node {
    struct Rectangle \{ float x1, y1, x2, y2; \};
    struct Rectangle Scale
    (struct Rectangle c, float h, float v) \{
        // your code here
        return big;
    \}
};

\hangindent-11cm \hangafter-9
\Problem Keep the center, scale the rectuangle
using `h' as horizontal multiplier and `v' as vertical multiplier.

\tikz [code block right] \node {
    structure Node \{
        double a;
        Node *b;
        Node *c;
    \}
};

\hangindent-5cm \hangafter-9
\Problem Reorder the following lines to make sense.
\yes `tail->next = new Node; tail = tail->next; tail->data = count++; tail->next = NULL;'
\yes `Node *head = new Node; head->data = count++; head->next = NULL;'
\yes `struct Node \{ int data; Node *next; \};'
\yes `Node *tail = head;'
\yes `int count = 0;'

\tikz [code block right] \node {
    string Cmplx2Strng (struct Cmplx z) \{
        string answer = to_string(z.a);
        answer += " + ";
        answer += to_string(z.b);
        answer += "i";
        return answer;
    \}
};

\hangindent-9cm \hangafter-9
\Problem Recall that $i = \sqrt{-1}$.
Assume that I use the code to the right to print complex numbers.
Guess the structure of `Cmplx'.

\hangindent-9cm \hangafter-9
\Problem When a stack is empty, what happens if we try to pop?
\no The stack will overflow.
\no The stack turns into a queue.
\no The stack enters an infinite loop.
\yes It depends on the implementation.

\no \tikz {
    \fill (0, 0) rectangle (1, 5);
    \fill [white] (0.1, 0.1) rectangle (0.9, 4.9);
    \foreach \y in {0,..., 4} {
        \pgfmathtruncatemacro\r{round(rnd*100)}
        \draw (0.2, \y+0.2) rectangle node {$\r$} (0.8, \y+0.8);
    }
}
\no \tikz {
    \fill (0, 0) rectangle (1, 5);
    \fill [white] (0.1, 0.1) rectangle (0.9, 5.1);
    \foreach \y in {0,..., 4} {
        \pgfmathtruncatemacro\r{round(rnd*100)}
        \draw (0.2, \y+0.2) rectangle node {$\r$} (0.8, \y+0.8);
    }
    \draw [->] (-1, 5) to[bend left=60] (0.3, 5);
    \draw [->] (0.7, 5) to[bend left=60] (2, 5);
    \path [->] (0.7, 0) to[bend right=60] (2, 0);
}
\no \tikz {
    \fill (0, 0) rectangle (1, 5);
    \fill [white] (0.1, -1) rectangle (0.9, 6);
    \foreach \y in {0,..., 4} {
        \pgfmathtruncatemacro\r{round(rnd*100)}
        \draw (0.2, \y+0.2) rectangle node {$\r$} (0.8, \y+0.8);
    }
    \draw [->] (-1, 5) to[bend left=60] (0.3, 5);
    \draw [->] (0.7, 0) to[bend right=60] (2, 0);
}
\no \tikz {
    \fill (0, 0) rectangle (1, 5);
    \fill [white] (0.1, -1) rectangle (0.9, 6);
    \foreach \y in {0,..., 4} {
        \pgfmathtruncatemacro\r{round(rnd*100)}
        \draw (0.2, \y+0.2) rectangle node {$\r$} (0.8, \y+0.8);
    }
    \draw [->] (-1, 5) to[bend left=60] (0.3, 5);
    \draw [->] (0.7, 5) to[bend left=60] (2, 5);
    \draw [->] (-1, 0) to[bend right=60] (0.3, 0);
    \draw [->] (0.7, 0) to[bend right=60] (2, 0);
}

\no \tikz {
    \foreach \y in {0,..., 5} {
        \begin{scope}[shift={(rnd/3, rnd/3)}]
            \pgfmathtruncatemacro\r{round(rnd*100)}
            \ifnum \y > 0
                \draw (3/4, \y+1/4) circle(1/10);
                \draw [->] (3/4, \y+1/4) to [out=-90, in=90, looseness=2] (prev);
            \fi
            \draw (0, \y) rectangle node {$\r$} (1/2, \y+1/2) rectangle (1, \y);
            \draw (0, \y+1/2+1/20) coordinate (prev);
        \end{scope}
    }
}
\no \tikz {
    \foreach \y in {0,..., 5} {
        \begin{scope}[shift={(rnd/3, rnd/3)}]
            \pgfmathtruncatemacro\r{round(rnd*100)}
            \ifnum \y > 0
                \draw (3/4, \y+1/4) circle(1/10) (prev2) circle(1/10);
                \draw [->] (3/4, \y+1/4) to [out=-90, in=90, looseness=2] (prev1);
                \draw [<-] (3/2, \y-1/20) to [out=-90, in=90, looseness=2] (prev2);
            \fi
            \draw (0, \y) rectangle node {$\r$} (1/2, \y+1/2) rectangle (1, \y) rectangle (3/2, \y+1/2);
            \draw (0, \y+1/2+1/20) coordinate (prev1) (5/4, \y+1/4) coordinate (prev2) ;
        \end{scope}
    }
}
\no \tikz {
    \begin{scope}[shift={(rnd/2, rnd/2)}, shift={(0, 0)}]
        \draw (0, 0) rectangle (1/2, 1/2) rectangle (1, 0) rectangle (3/2, 1/2);
        \draw (0, -1/20) coordinate (A);
        \pgfmathtruncatemacro\r{round(rnd*100)}
        \node at (1/4, 1/4) {$\r$};
    \end{scope}
    \begin{scope}[shift={(rnd/2, rnd/2)}, shift={(2, 0)}]
        \draw (0, 0) rectangle (1/2, 1/2) rectangle (1, 0) rectangle (3/2, 1/2);
        \draw (0, -1/20) coordinate (B);
        \pgfmathtruncatemacro\r{round(rnd*100)}
        \node at (1/4, 1/4) {$\r$};
    \end{scope}
    \begin{scope}[shift={(rnd/2, rnd/2)}, shift={(4, 0)}]
        \draw (0, 0) rectangle (1/2, 1/2) rectangle (1, 0) rectangle (3/2, 1/2);
        \draw (0, -1/20) coordinate (C);
        \pgfmathtruncatemacro\r{round(rnd*100)}
        \node at (1/4, 1/4) {$\r$};
    \end{scope}
    \begin{scope}[shift={(rnd/2, rnd/2)}, shift={(6, 0)}]
        \draw (0, 0) rectangle (1/2, 1/2) rectangle (1, 0) rectangle (3/2, 1/2);
        \draw (0, -1/20) coordinate (D);
        \pgfmathtruncatemacro\r{round(rnd*100)}
        \node at (1/4, 1/4) {$\r$};
    \end{scope}
    \begin{scope}[shift={(rnd/2, rnd/2)}, shift={(1, -2)}]
        \draw (0, 0) rectangle (1/2, 1/2) rectangle (1, 0) rectangle (3/2, 1/2);
        \draw (0, -1/20) coordinate (E);
        \pgfmathtruncatemacro\r{round(rnd*100)}
        \node at (1/4, 1/4) {$\r$};
        \draw (3/4, 1/4) circle (0.1);
        \draw [->] (3/4, 1/4) to [out=90, in=-90, looseness=2] (A);
        \draw (5/4, 1/4) circle (0.1);
        \draw [->] (5/4, 1/4) to [out=90, in=-90, looseness=2] (B);
    \end{scope}\begin{scope}[shift={(rnd/2, rnd/2)}, shift={(5, -2)}]
        \draw (0, 0) rectangle (1/2, 1/2) rectangle (1, 0) rectangle (3/2, 1/2);
        \draw (0, -1/20) coordinate (F);
        \pgfmathtruncatemacro\r{round(rnd*100)}
        \node at (1/4, 1/4) {$\r$};
        \draw (3/4, 1/4) circle (0.1);
        \draw [->] (3/4, 1/4) to [out=90, in=-90, looseness=2] (C);
        \draw (5/4, 1/4) circle (0.1);
        \draw [->] (5/4, 1/4) to [out=90, in=-90, looseness=2] (D);
    \end{scope}
    \begin{scope}[shift={(rnd/2, rnd/2)}, shift={(3, -4)}]
        \draw (0, 0) rectangle (1/2, 1/2) rectangle (1, 0) rectangle (3/2, 1/2);
        \pgfmathtruncatemacro\r{round(rnd*100)}
        \node at (1/4, 1/4) {$\r$};
        \draw (3/4, 1/4) circle (0.1);
        \draw [->] (3/4, 1/4) to [out=90, in=-90, looseness=2] (E);
        \draw (5/4, 1/4) circle (0.1);
        \draw [->] (5/4, 1/4) to [out=90, in=-90, looseness=2] (F);
    \end{scope}
}

\Problem Which drawing represents a fixed-length array?

\Problem Which drawing represents a stack (LIFO)?

\Problem Which drawing represents a queue (FIFO)?

\Problem Which drawing represents a double-ended queue?

\Problem Which drawing represents a (singly) linked list?

\Problem Which drawing represents a doubly linked list?

\Problem Which drawing represents a binary tree?

\section{Algorithm}

\tikz [code block right] \node {
    int attempts = ???;
    int guess = 0;
    int target = rand() \% 100;
    while (attempt > 0) \{
        attempt -= 1;
        cin >> guess;
        if (guess > target) \{ cout << "Too high"; \}
        else if (guess < target) \{ cout << "Too low"; \}
        else \{ cout << "Win"; return; \}
    \}
    cout << "Lose: no more attempts";
};

\hangindent-13cm \hangafter-99
\Problem Consider the program to the right.
You want to make it as hard as possible to a beginner,
but you also want to ensure that you yourself can always win.
What should you set `attempts' to?
\no 1
\no 2
\no 3
\no 4
\no 5
\no 6
\yes 7
\no 8
\no 9
\no 10
\no 11
\no 12
\no 13
\no 14
\no 15
\no 16
\no 17
\no 18
\no 19
\no 20

\tikz [code block right] \node {
    void BreadthOfTheField(int A[], int a, int b) \{
        while (a < b) \{
            int t = A[a]; A[a] = A[b - 1]; A[b - 1] = t;
            BreadthOfTheField(A, a + 1, b - 1);
            break;
        \}
    \}
};

\hangindent-13cm \hangafter-5
\Problem What is the purpose of the function to the right?
\no To sort an array
\no To rotate an array
\yes To revert an array
\no To shuffle an array
\no To summarize an array

\tikz [code block right] \node {
    void TrickOfTheKitchen(int A[100], int c) \{
        int B[100];
        for (int i = 0; i < 100; i++) \{
            B[i] = A[(i + c) \% 100];
        \}
        for (int i = 0; i < 100; i++) \{
            A[i] = B[i];
        \}
    \}
};

\hangindent-10cm \hangafter-9
\Problem Consider another function as show to the right.
This is my first implementation that uses a second array `B'.
I want to achieve the same effect,
but without using extra memory (i.e., the second array `B').
So, instead, I call `BreadthOfTheField' three times.
With what parameters should I call it?
\blank{(A, 0, c); (A, c, 100); (A, 0, 100);}

\immediate\closeout\answer

\clearpage

\section*{Computer Programming - Midterm Mock Exam - Answer Sheet}

\tikzset{
    every picture/.style={
        baseline=-3pt,
        blue!70!black,
        every node/.style={font=\fontsize{24pt}{0}\fontspec{Comic Sans MS}}
    },
}

Name (zh or en) \tikz \node {Hsin-Po Wang}; \hfil
Student ID \tikz \node {B00201054};


\input{\jobname.answer}

\end{document}