
\documentclass{beamer}
\begin{document}
\begin{frame}
    \frametitle{Poincaré conjecture}
    \begin{itemize}
        \item Every simply connected, closed 3-manifold is homeomorphic to the 3-sphere.
        \item Proven by Grigori Perelman in 2003.
        \item Uses Richard S. Hamilton's Ricci flow.
    \end{itemize}
\end{frame}

\begin{frame}
    \frametitle{Birch and Swinnerton-Dyer conjecture}

    \begin{itemize}
        \item <1-> Predicts a relationship between the rank of the group
            of rational points and the behavior of the L-function at \(s = 1\).
        \item <2-> Iwasawa theory with Euler systems (Kolyvagin--Rubin)
            is currently the most promising approach.
        \item <3-> Profound impact on number theory and cryptography;
            would clarify ranks of elliptic curves
            and reshape algorithms and security assumptions.
    \end{itemize}
\end{frame}

\begin{frame}
    \frametitle{Hodge conjecture}

    \begin{itemize}
        \item <+-> States that certain de Rham cohomology classes
            are algebraic.
        \item <+-> Motivic cohomology and derived algebraic geometry
            (Voevodsky--Deligne program) offer potential pathways.
        \item <+-> Would unify topology and algebraic geometry;
            enable algorithmic methods for classifying algebraic varieties
            and deepen understanding of the structure of complex manifolds.
    \end{itemize}
\end{frame}

\begin{frame}
    \frametitle{Navier–Stokes existence and smoothness}

    \begin{itemize} [<+->]
        \item Governs the motion of fluid substances.
        \item The existence and smoothness of solutions
            in three dimensions is unknown.
        \item Most promising method: convex integration for
            constructing wild weak solutions (De Lellis--Székelyhidi), and
            scale-critical regularity via harmonic analysis (Koch--Tataru).
    \end{itemize}
\end{frame}

\begin{frame}
    \frametitle{P versus NP}

    \begin{itemize} [<+- | alert@+>]
        \item Asks whether every efficiently verifiable problem
            is also efficiently solvable.
        \item Foundational in complexity theory;
            impacts optimization, cryptography, and AI.
        \item Key lines of attack: circuit lower bounds,
            barriers from natural proofs (Razborov--Rudich),
            and algebraic/geometric complexity (VP vs VNP).
        \item Consequences: if P $=$ NP, many cryptosystems collapse;
            if P $\neq$ NP, formal hardness underpins
            secure cryptography and explains intractability.
    \end{itemize}
\end{frame}
\end{document}

