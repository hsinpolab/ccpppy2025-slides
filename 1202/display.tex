
\documentclass{article}
\usepackage{mathtools} % mathtools includes amsmath
\begin{document}

    Use \verb|\[ ... \]|.
    Do not use \verb|$$ ... $$|, it is too old.
    \[ a^2 + b^2 = c^2. \]
    Put punctuation in the display math too.

    The \verb|equation*| environment from amsmath package
    is the same as \verb|\[ ... \]|.
    \begin{equation*}
        e^{i\pi} + 1 = 0.                                         \tag{Euler}
    \end{equation*}
    Use \verb|\tag{...}| to name or explain the equation.

    The \verb|align*| environment from amsmath package
    is for multiple equations that allow you to align verbs.
    (Each equal sign is a verb.)
    \begin{align*}
        f(x) &= (x + 1)^2 \\
            &= x^2 + 2x + 1.                                    \tag{Expand}
    \end{align*}
    The ampersand \verb|&| is the alignment point.
    The punctuation is needed for the last line.

    Use \verb|gather*| to center multiple equations without alignment.
    \begin{gather*}
        V - E + F = 2,                                         \tag{Euler} \\
        E = mc^2.                                              \tag{Einstein}
    \end{gather*}
    note that each line has a punctuation.

\end{document}

