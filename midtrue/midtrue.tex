% !TEX encoding = UTF-8
% !TEX program = lualatex

\documentclass[a4paper]{article}

\usepackage[left=1.5cm, right=1.5cm, top=1cm, bottom=2cm]{geometry}

\usepackage{mathtools, amssymb}

\usepackage{fontspec}
    \setmainfont{Noto Serif}
    \setsansfont{Noto Sans}
    \setmonofont{Noto Sans Mono}

\usepackage{tikz}

\makeatletter
\tikzset{
    every picture/.style={baseline=(current bounding box)},
    code block/.style={
        fill=gray!10, align=left, inner ysep=0pt,
        font={\hskip-20pt\ttfamily\catcode32=12\catcode13=13\relax}
    },
    code block right/.style={
        overlay, shift={(18, 0.3)}, 
        every node/.style={code block, below left},
    }
}
\makeatother

\newcounter{p}
\def\Problem#1{%
    \stepcounter{p}[\arabic{p}]\hskip0pt plus2em
    \immediate\write\answer{\hbox to#1{[\arabic p]\hss}\hskip0pt plus#1}%
}
\newcounter{c}
\def\yes{\no}
\def\no{\stepcounter c\ifnum\value c=27\setcounter c1\fi(\Alph c) }
\def\blank#1{}
\newwrite\answer
\immediate\openout\answer=\jobname.answer

\begin{document}

\parskip1em
\parindent0pt

\catcode`\#=12
\catcode`\%=12
\catcode`\^=12
\catcode`\&=12
\catcode`\_=12
\catcode`\'=13 \def'{\rmfamily}
\catcode`\`=13 \def`{\ttfamily}
\catcode13=13 \def
{\\} \catcode13=5\relax

\section*{Computer Programming - Midterm Mock Exam - Question Sheet}

\fontspec{Noto Serif TC}

考試時間 Test time 15:50 pm to 18:10 pm; 2.5 hours.
以投影時鐘為準 Using the projected clock as official time.
在答案卷上作答 Answer on the answer sheet.
每題一分 One point per problem.
全對才給分 No partial credit.
不倒扣 No penalty for wrong answers.
禁止外部資源(比照學測、分科) No external resources, like TOEFL.

\rmfamily

\tikz [code block right] \node {
    \#include <iostream>
    using namespace std;
    int main() \{
        cout << "hello"
        << " " " " "ntu";
    \}
};

\hangindent-5cm \hangafter-9
\Problem{5em}
The code to the right outputs
\no `hellontu'
\no `hello~ntu'
\yes `hello~~ntu'
\no compile error, no output

\hangindent-5cm \hangafter-9
\Problem{5em}
Regarding string literals, what is true?
\no It's the only way to define strings in C++.
\no They are not stored in the executable file.
\no They can only contain alphabetic characters.
\yes They are concatenated even without a plus sign.

\Problem{5em}
Why C++ libraries separate header files and implementation files
and only the header files are included?
Select two that apply.
\yes The implementation is already compiled.
\yes It slows down compilation if reading both.
\no To prevent errors when compiling implementation files.
\no It is the user's job to implement the functions declared in headers.

\Problem{5em}
C++ is invented in
\yes 1983
\no 1993
\no 2003
\no 2013

\tikz [code block right] \node {
    int main (int argc, char *argv[]) \{
        return 0;
    \}
};

\hangindent-8cm \hangafter-3
\Problem{5em} Which are correct about `main'?
\yes A cpp file do not have to have a `main' function.
\yes For `g++-14', `argv' may contain things like `-o' and `-Wall'.
\no It is required to return a nonzero value when something goes wrong.
\no It is required to return zero when the program finishes successfully.

\tikz [code block right] \node {
    g++-14 -Wall option_pricing -o op
    ./op <<< "100 0.05 0.2 1 100"
};

\hangindent-8cm \hangafter-2
\Problem{5em} Select the correct option.
\no `-o' specifies the optimization level.
\no `option_pricing' enables on the pricing option.
\no `g++-14' refers to the compiler version announced in 2014.
\yes `-Wall' turns on every warning provided by the compiler.

\Problem{5em} Which command line create a new folder?
\no `ls'
\no `cat'
\no `touch'
\yes `mkdir'

\Problem{5em} Which command is used in the jupyter notebook
to output the cell content to a cpp file?
\no `ls'
\yes `cat'
\no `touch'
\no `mkdir'

\Problem{5em}
`T' is a type and `T A[100];' is an array.
What is `&A[11] - &A[10]'?
\yes 1
\no the size of `T' in bits
\no the size of `T' in bytes
\no compiler error, as `(&A)' is not an array and `(&A)[11]' is meaningless

\Problem{4em} The binary representation of `int16_t a = -10101' is?

\tikz [code block right] \node {
    00111111100000000000000000000000 = 1
    01000000000000000000000000000000 = 2
    01000000010000000000000000000000 = 3
    01000000100000000000000000000000 = 4
    01000000101000000000000000000000 = 5
    01000000110000000000000000000000 = 6
    01000000111000000000000000000000 = 7
    01000001000000000000000000000000 = 8
    01000001000100000000000000000000 = 9
    01000001001000000000000000000000 = 10
};

\Problem{15em}
What is the binary representation of $255$?

\Problem{15em}
What is the binary representation of $-0.375$?

\Problem{15em}
What is the binary representation of $2\sp{100}$?

\Problem{15em}
What is the binary representation of $2\sp{100} + 1$?

\hangindent-9cm \hangafter-9
\Problem{5em}
What standard is that?
\yes IEEE 754
\no IEEE 7382
\no IEEE 1111
\no IEEE 1.414

\Problem{5em}
What is the `sizeof()' of this data type
\yes 4
\no at least 4
\no 32
\no at least 32

\Problem{5em}
The largest number it can represent is $3.4 \times 10\sp{\text{what}}$?

\Problem{5em}
Knowing that $\log\sb2(10) = 3.322$,
what is the best way to define `PI' in this data type?
\no `this_type PI = 3.14159;'
\yes `this_type PI = 3.1415926;'
\no `this_type PI = 3.1415926535;'
\no `this_type PI = 3.141592653589;'

\Problem{5em}
Which are testing if a positive integer `n' is a multiple of 8?
\yes `n % 8 == 0'
\yes `n \& 7 == 0'
\no `n & (n - 1) >= 8'
\yes `n / 8 == (n + 7) / 8'

\tikz [code block right] \node {
    bool isSquare (int n) \{
        // your code here
        // your code here
        return false;
    \}
};

\hangindent-6cm \hangafter-9
\Problem{15em}
Finish this function that checks whether `n' is a perfect square.
You can assume that `n' is between `0' and `1000'.

\Problem{5em}
What is \verb"'^'^'^'"?

\tikz [code block right] \node {
    float f = 1;
    float dfdt = 0;
    float dt = 1e-3;
    for (float t = 0; t < 10; t += dt) \{
        float ddfddt = f;
        dfdt += ddfddt * dt;
        f += dfdt * dt;
    \}
};

\hangindent-9cm \hangafter-9
\Problem{5em} Some numerical ODE code is presented to the right.
What is the equation being solved?
\no $f\sp{\prime} = f$
\yes $f\sp{\prime\prime} = f$
\no $f\sp{\prime} = -f$
\no $f\sp{\prime\prime} = -f$

\hangindent-9cm \hangafter-9
\Problem{5em} Select the implied initial conditions.
\no $f(0) = 0$
\yes $f(0) = 1$
\yes $f\sp\prime(0) = 0$
\no $f\sp\prime(0) = 1$

\hangindent-9cm \hangafter-2
\Problem{5em} Select all that apply.
\yes This is called the Euler method.
\yes In industry, there are better methods.
\no The loop runs for about 100000 iterations.
\no `t' is casted into an `int' before comparing to `10'.

\hangindent-9cm \hangafter-1
\Problem{5em} Select the true statement.
\yes `break' will break the inner-most loop.
\no `continew' will skip to the next iteration of the inner-most loop.
\no `fbreak' will break for-loop and ignore while-loop.
\no `wbreak' will break while-loop and ignore for-loop.

\tikz [code block right] \node {
    int S = 0;
    for (int i = 0; i < 20000000; i++) \{
        for (int j = 0; j < 20000000; j++) \{
            if (rand() % 17) \{ S++; \}
        \}
    \}
};

\hangindent-10cm \hangafter-9
\Problem{5em} 
How long does it take to run the loop above?
\no about 1--10 microseconds
\yes about 1--10 seconds
\no about 1--10 days
\no about 1--10 years

\hangindent-10cm \hangafter-2
\Problem{5em}
When we have nested for loop like above,
what should we do to monitor its progress?
\no Add `cout << t' in the inner loop.
\no Add `cout << t' in the outer loop.
\yes print a dot for every one thousand iterations.
\no Use a debugger to manually step through the code.

\Problem{5em}
Select the two correct declarations so that we can use `A[j][j]'
to access the entry of a matrix `A' of size 100x200.
\yes `int A[100][200] = \{\};'
\no `int A = new int[100][200];'
\no `int **A = new int[100][200];'
\yes `int (*A)[200] = new int[100][200];'

\tikz [code block right] \node {
    #include <iostream>
    #include <random>
    using namespace std;
    int main () \{
        mt19937 rand(20251022);
        mt19937 rane(20251022);
        mt19937 ranf(20251022);
        cout << rand() << rand() << rand() << endl;
        cout << rane() % 16 << endl;
        cout << ranf() % 16 << endl;
        cout << rane() * ranf() % 16 << endl;
        cout << ranf() * rane() % 16 << endl;
    \}
};

\Problem{5em} The first line of the output is \\
`21364248221561776959936401120'. \\
What is the second line of output?

\Problem{5em} What is the third line of output?

\Problem{5em} What is the fourth line of output?

\Problem{5em} What is the fifth line of output?

\hangindent-11cm \hangafter-9
\Problem{15em}
Use `a', `b', `c', `d', `x', `add(, )', and `mul(, )'
to express cubic polynomial $ax\sp3 + bx\sp2 + cx + d$.
Use the minimum number of `add' and `mul'.

\tikz [code block right] \node {
    float i(float x) \{ return 10 * x + 1; \}
    float i(int x) \{ return 10 * x + 2; \}
    int f(float x) \{ return 10 * x + 3; \}
    int f(int x) \{ return 10 * x + 4; \}
};

\hangindent-10cm \hangafter-9
\Problem{5em} What is `f(f(i(5)))'?

\Problem{5em} What is `f(f(i(i(f(f(i(5.f)))))))'?

\tikz [code block right] \node {
    int a = 1, b = 3, c = 2;
    int *p = &a, *q = &b, *r = &c;
    int **x = &p, **y = &q, **z = &r;
    cout << **x << **y << **z << endl;
    swap(x, y);
    cout << **x << **y << **z << endl;
    swap(a, c); swap(q, r);
    cout << **x << **y << **z << endl;
    swap(*x, *z);
    cout << **x << **y << **y << endl;
    swap(*p, *q); swap(**y, **z);
    cout << **x << **y << **z << endl;
};

\hangindent-8cm \hangafter-9
\Problem{5em}
For the code to the right, what is the first line of output?

\Problem{5em}
What is the second line of output?

\Problem{5em}
What is the third line of output?

\Problem{5em}
What is the fourth line of output?

\Problem{5em}
What is the fourth line of output?

\Problem{5em} `1*a**p***x' is?

\hangindent-8cm \hangafter-1
\Problem{5em}
A private member function
\no can be called by any function (even outside the class)
\yes can be called by other member functions of the same class
\no cannot access private member variables
\no cannot access public member variables

\tikz [code block right] \node {
    struct Circle \{ float x, y, r; \};
    void Shift (struct Circle& c, float x, float y)
        \{/*your code here*/\}
    void Rotate (struct Circle& c, float r)
        \{/*your code here*/\}
};

\hangindent-11cm \hangafter-9
\Problem{15em}
Finish the function that shifts the center of the circle by `(x, y)'.
(Note that `c' is passed by reference so you can just modify `c'.)

\Problem{15em}
Finish the function that rotates the circle by `r' radians
(the default unit for trigonometric functions) about the origin.

\tikz [code block right] \node {
    int A[100];
    /*computation for A*/
    /*prepare-before*/
    for (int i = 0; i < 100; i++) \{ /*in-loop*/ \}
};

\hangindent-11cm \hangafter-9
\Problem{15em}
I want to compute the maximum `maxA' of an integer array `A' of length `100'.
What should `/*prepare-before*/' be

\Problem{15em}
What should `/*in-loop*/' be?'

\Problem{15em}
If, instead of maximum,
I want to check if at least 5 elements of `A' are greater than `10',
what should `/*prepare-before*/' be?

\Problem{15em}
What should `/*in-loop*/' be?

\Problem{15em}
After the loop, I should let `bool check5gt10 = '?

\Problem{5em}
C\# (where \# is pronounced `sharp' and stands for four plus signs)
\no 1970
\no 1980
\no 1990
\yes 2000

\immediate\closeout\answer

\clearpage

\hbox{}

\clearpage

\section*{Computer Programming - Midterm Mock Exam - Answer Sheet}

Name (zh or en) \hfil Student ID

\baselineskip1em plus1fil

\noindent
\input{\jobname.answer}

\end{document}